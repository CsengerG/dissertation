\documentclass[class=article, crop=false]{standalone}
\usepackage[utf8]{inputenc}
\usepackage{hyperref}
\usepackage{comment}

\usepackage[margin=1in]{geometry}

\begin{document}

The project was a success. My success criterion was to implement a 
syntax-level interpreter for Eff, to design a low-level byte code for Eff,
to implement a compiler which can transform Eff programs into a linearised
representation and finally to implement a virtual machine which is capable of
interpreting the designed byte code. Furthermore, the above components had to be
reasonably efficient when compared to existing solutions.

The work described above was carried out and all components were successfully
implemented. Both my CEK interpreter and my SHADE VM outperformed Eff on various
benchmarks and the performance of SHADE VM was often comparable to that of
Multicore OCaml.

\paragraph{Travel Diary}

This dissertation is actually a story about a journey
which starts in the theoretical valleys of Mathematics, leads through the more
practical (but still quite theoretical) areas of Computer Science and finally
ends in the realms of practical Systems Programming.

This work was partly motivated by the desire to explore how algebraic effect
handlers could be used in systems programming. The use of algebraic effect
handlers is appealing as they provide a convenient abstraction for the handling
of side effects functionally. However, efficient runtime systems often work
with low-level bytecodes rather than with syntax-level representations of
programs. Designing appropriate bytecodes and ways to realise algebraic effect
handlers closer to the hardware is necessary if we are ever
going to use them in the real world. The existence of Multicore OCaml and SHADE VM
proves that this is possible. To show a real world systems programming application
I exhibited a simple concurrent system (a toy webserver) which was executed
in the runtime system I designed. 

\section{Further work}

\paragraph{Optimisations}
The fact that there are algebraic theories behind algebraic effects and their
handlers allows us to prove certain compiler transformations correct by
equational reasoning. This gives us the opportunity to take a set of effect
equations (for instance, the equation $\mathtt{assign}(x,y); \mathtt{assign}(x,z) =
\mathtt{assign}(x, z)$ would show that the first \texttt{assign} can be optimised
away) and generate new optimisations (perhaps entirely new ones no-one has
thought about yet or ones that are dependent on the semantics of some effects).

\paragraph{Type and effect systems}
\cite{bauer2013effect} allow for even more optimisations
and the opportunity to parallelise independent parts of programs. As hardware is
getting increasingly parallel such optimisations can help us to get the most out
of our multicore machines.

\paragraph{Open source}
The project is packaged up as an OPAM package and I intend to make it available
on Github for other programming language enthusiasts.

\paragraph{Part II project}
My implementation was concerned with only the core
features of Eff. There would be countless ways to improve the compiler or the
virtual machine. The rules of SHADE VM are simple enough that a byte code interpreter
for SHADEcode should not be too hard to implement in a low-level language such as C.
However, this would have to involve designing a custom garbage collector for SHADE
which might prove interesting.

\paragraph{Masters or PhD}
As the size of continuations is not too big it might make sense to think about
these control operators in the context of distributed systems too.
For instance, imagine the following situation where the communication overhead
in the network might be reduced.

Alice and Bob are communicating. Alice sends a message $M_1$ to Bob and Bob
replies with $M_2$. Based on the contents of $M_2$ Alice performs some kind of
computation $c$ and replies with $M_3$. If $c$ can be realised as a continuation
resumed with $M_2$ as its argument and given that shadows/continuations can be
serialised using a convenient byte code representation, would it make sense to
\emph{package up a decision as a continuation} and attach it to the original
message $M_1$? This would eliminate the need to exchange $M_2$ and $M_3$
over the network as Bob could resume the continuation using $M_1$ and obtain
the result \emph{locally}.


\section{Self-reflection}

If I were to start the project again I would try to be less ambitious. I vastly
underestimated the intellectual challenge behind continuations, handlers and
control operators.

At the beginning of the project I got a bit carried away with syntactic sugar
and trying to implement a lot of features of the language properly. This proved
to be counterproductive and I had to abandon these efforts. If I had to do this
again I would make the development even more iterative than it was: I would
design a smaller minimal viable product first, build a very minimal but working
(and tested!) pipeline and then iterate more on improvements. This was my
strategy when developing, but I could do it better the next time.

\paragraph{Closing thoughts}

Furthermore, I think there is a lack of elementary material concerning this
topic on the web. I am hoping that my Preparation chapter and this dissertation
as a whole will in some ways help popularise algebraic effects and their
handlers and that readers of this document will find them less cryptic than I
first did.

\end{document}
