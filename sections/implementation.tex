\documentclass[class=article, crop=false]{standalone}
\usepackage[subpreambles=true]{standalone}
\usepackage{import}
\usepackage{ebproof}
\usepackage[utf8]{inputenc}
\usepackage{tikz}
\usepackage{hyperref}
\usepackage{amsmath}
\usepackage{amssymb}
\usepackage{listings}
\usepackage{verbatim}
\usepackage{a4wide}
\usepackage[super]{nth}
\usepackage{float}
\usepackage{subcaption}

\ifstandalone
\lstdefinelanguage{efflang}
{
    % list of keywords
    morekeywords={
        let,
        perform,
        continue,
        val,
        effect,
        in,
        if, then, else,
        with, handle, handler,
        finally,
        match
    },
    sensitive=false,
    morecomment=[s]{(*}{*)},
    morestring=[b]"
}

% Define Colors
\usepackage{color}
\definecolor{eclipseBlue}{RGB}{42,0.0,255}
\definecolor{eclipseGreen}{RGB}{63,127,95}
\definecolor{eclipsePurple}{RGB}{127,0,85}
\definecolor{bg}{RGB}{248,248,248}

\lstset{
    language={efflang},
    basicstyle=\small\ttfamily, % Global Code Style
    captionpos=b, % Position of the Caption (t for top, b for bottom)
    extendedchars=true, % Allows 256 instead of 128 ASCII characters
    tabsize=2, % number of spaces indented when discovering a tab 
    columns=fixed, % make all characters equal width
    keepspaces=true, % does not ignore spaces to fit width, convert tabs to spaces
    showstringspaces=false, % lets spaces in strings appear as real spaces
    breaklines=true, % wrap lines if they don't fit
    %frame=trbl, % draw a frame at the top, right, left and bottom of the listing
    %frameround=tttt, % make the frame round at all four corners
    framesep=4pt, % quarter circle size of the round corners
    numbers=left, % show line numbers at the left
    numberstyle=\tiny\ttfamily, % style of the line numbers
    commentstyle=\color{eclipseGreen}, % style of comments
    keywordstyle=\color{eclipsePurple}, % style of keywords
    stringstyle=\color{eclipseBlue}, % style of strings
    xleftmargin=.05\textwidth,
    framexleftmargin=17pt,
    backgroundcolor=\color{bg}
    %xrightmargin=.2\textwidth
}
\else
\lstdefinelanguage{efflang}
{
    % list of keywords
    morekeywords={
        let,
        perform,
        continue,
        val,
        effect,
        in,
        if, then, else,
        with, handle, handler,
        finally,
        match
    },
    sensitive=false,
    morecomment=[s]{(*}{*)},
    morestring=[b]"
}

% Define Colors
\usepackage{color}
\definecolor{eclipseBlue}{RGB}{42,0.0,255}
\definecolor{eclipseGreen}{RGB}{63,127,95}
\definecolor{eclipsePurple}{RGB}{127,0,85}
\definecolor{bg}{RGB}{248,248,248}

\lstset{
    language={efflang},
    basicstyle=\small\ttfamily, % Global Code Style
    captionpos=b, % Position of the Caption (t for top, b for bottom)
    extendedchars=true, % Allows 256 instead of 128 ASCII characters
    tabsize=2, % number of spaces indented when discovering a tab 
    columns=fixed, % make all characters equal width
    keepspaces=true, % does not ignore spaces to fit width, convert tabs to spaces
    showstringspaces=false, % lets spaces in strings appear as real spaces
    breaklines=true, % wrap lines if they don't fit
    %frame=trbl, % draw a frame at the top, right, left and bottom of the listing
    %frameround=tttt, % make the frame round at all four corners
    framesep=4pt, % quarter circle size of the round corners
    numbers=left, % show line numbers at the left
    numberstyle=\tiny\ttfamily, % style of the line numbers
    commentstyle=\color{eclipseGreen}, % style of comments
    keywordstyle=\color{eclipsePurple}, % style of keywords
    stringstyle=\color{eclipseBlue}, % style of strings
    xleftmargin=.05\textwidth,
    framexleftmargin=17pt,
    backgroundcolor=\color{bg}
    %xrightmargin=.2\textwidth
}
\fi

\ifstandalone
\renewcommand{\leadsto}{\rightsquigarrow}
\newcommand{\dmid}{\ \parallel \ }

\newcommand{\effFalse}{\mathbf{false}}
\newcommand{\effTrue}{\mathbf{true}}
\newcommand{\effLeft}{\mathbf{Left}\ }
\newcommand{\effRight}{\mathbf{Right\ }}
\newcommand{\effFun}{\mathbf{fun}\ }
\newcommand{\effHandler}{\mathbf{handler}\ }
\newcommand{\effVal}{\mathbf{val}\ }
\newcommand{\effWith}{\mathbf{with}\ }
\newcommand{\effHandle}{\ \mathbf{handle}\ }
\newcommand{\effIf}{\mathbf{if}\ }
\newcommand{\effThen}{\ \mathbf{then}\ }
\newcommand{\effElse}{\ \mathbf{else}\ }
\newcommand{\effAbsurd}{\mathbf{absurd}\ }
\newcommand{\effMatch}{\mathbf{match}\ }
\newcommand{\effLet}{\mathbf{let}\ }
\newcommand{\effIn}{\ \mathbf{in}\ }
\newcommand{\effRec}{\mathbf{rec}\ }
\newcommand{\effEffect}{\mathbf{effect}\ }
\newcommand{\effFinally}{\mathbf{finally}\ }
\newcommand{\effOp}{\mathtt{op}}
\newcommand{\effPerform}{\mathbf{perform}\ }
\newcommand{\tto}{\twoheadrightarrow}

\newcommand{\handlerType}{\Rightarrow}
\newcommand{\boolType}{\mathtt{bool}}
\newcommand{\unitType}{\mathtt{unit}}
\newcommand{\emptyType}{\mathtt{empty}}

\newcommand{\defEq}{\stackrel{\text{def}}{=}}

\newcommand{\cek}[1]{\langle #1 \rangle}
\newcommand{\secd}[1]{\langle #1 \rangle}
\fi

\begin{document}

This chapter is concerned with the implementation of a CEK interpreter for Eff and
a SECD-like virtual machine for Eff. As the project is theory heavy, the first
half of the chapter is about the theory that had to be developed and understood
to implement the parts described in the second half.

\section{Frontend -- Parsing and the IR}

<code example here where val is weird>

First De Bruijn indices on syntax tree.
Match is not supported but we have let (x,y) =
atomic patterns
continue k is rewritten to function application

Then delete the expression/computation distinction and convert into
an intermediate representation.

The intermediate representation is in A-normal form 



\begin{itemize}
\item Concrete syntax differs slightly from abstract syntax
\item De Bruijn indices
\begin{itemize}
    \item Local environment
    \item Global environment
\end{itemize}
\item LetArgs --> Fun(Fun(...)) -- Something similar with sums and tuples and variants
\item LetRec --> RecFun ...
\item IR is ANF-like -- Makes the compilation ``easy''
\end{itemize}

\section{Compilation and back-ends}

\subsection{CEK abstract machine}

Links does CEK (Hillerstrom) and it is possible to do this with Eff --
this section presents a similar CEK machine to the one Hillerstrom describes (cite)

\subsubsection{Closures and hlosures}

example where handler has free variables

\begin{itemize}
    \item Hillerstrom Master dissertation.
    \item Closures, hlosures, delimited continuations, concatenation of continuations
\end{itemize}

\begin{itemize}
    \item Type in rules
\end{itemize}

\subsection{CEK implementation}

In OCaml, the implementation closely follows the transition rules of the abstract machine.

We represent a CEK state with the type \verb|cek_state|:
\begin{verbatim}
    type cek_state = {
        c : cek_control;
        e : cek_environment;
        k : cek_hlosure_frame list;
    }
\end{verbatim}

The most interesting is the \verb|cek_hlosure_frame list|:
\begin{verbatim}
    type cek_hlosure_frame = {
        hlosure : cek_hlosure;
        kont_stack : cek_closure list;
    }
\end{verbatim}

\subsection{The SHADE virtual machine}

The SHADE virtual machine is a SECD-like virtual machine which was the goal of
my project.
The SHADE name is an anagram from the initials A,D,E,H,S -- the main
components of the machine.
It had many versions and the final version turned out to be surprisingly similar
to the solution Multicore OCaml uses with its fibres -- the heap-allocated fibres
of OCaml show a lot of resemblance with the shadows of the SHADE machine.

The SHADE virtual machine interprets SHADE byte code which will be described
in the next section. Here I give the description of the machine with its transition
rules.

The state of the machine at any point can be characterised with a two-tuple
$$ \langle A, D \rangle $$
where $A$ is the accumulator and $D$ is a list of \emph{shadows}, essentially a stack
of continuations like the dump in SECD.
A shadow is a five-tuple $$ \langle pc, E, H, S \rangle $$, where $pc$ is the
program-counter of the shadow, $E$ is an environment, $H$ is a hlosure and $S$ is
a stack.

A transition of the SHADE machine is determined by program counter, the accumulator
and the state of the \emph{top-shadow}. To avoid having to refer to the top-shadow
using D I use the notation $\langle A, d, D \rangle$ to mean $\langle A, d :: D \rangle$.
In the program counter field ($pc$) I will write the instruction the program counter
points to rather than its value.

The figure is only dealing with inter-shadow instructions. Intra-shadow instructions
are standard and are described in the next section.

\begin{figure}
    Initialisation:
    $$\text{program} \to \shade{\shUnit, (0, \{\}, H_{def}, []), []}$$
    Termination:
    $$ \shade{v, (\shHalt, E, H, S), D} \to v $$
    Transition rules:
    $$ \shade{closure(cp, e), (\shCast, E, H, h :: S), D} \to \shade{A, (cp, e, h, []), (pc+1, E, H, S) :: D} $$
    $$ \shade{A, (\shRett, E, H, S), d :: D} \to \shade{A, d, D} $$
    $$ \shade{A, (\shApply, E, H, closure(cp, e) :: S), D} \to \shade{A, (cp, e, H, (pc, E) :: S), D} $$
    $$ \shade{A, (\shApply, E, H, cont (\kappa, K) :: S), D} \to \shade{A, \kappa \| K @ ((pc, E, H, S) :: D)} $$
    $$ \shade{A, (\shHandle} $$
    $$ \shade{A, (\shEndHandle} $$
    $$ \shade{A, (\shFin} $$
    $$ \shade{A, (\shThrow} $$

    \caption{SHADE machine transition rules}
    \label{shade-machine}
\end{figure}

\subsection{The SHADE instruction set}

This section

\begin{table}
    \centering
\begin{tabular}{|l|l|l|l}
    Instruction & Description & Action \\ \hline \hline
    \vmPush & Push the accumulator to the stack & S := A :: S \\ \hline
    \vmPop  & Remove the top element of the stack & S := tail S \\ \hline
    \vmAcc{n} & Copy the $n$th element of the stack in the accumulator & A := S[n] \\ \hline
    const & & \\ \hline
    halt & & \\ \hline
    jump & & \\ \hline
    branchifnot & & \\ \hline
    apply & & \\ \hline
    ret & & \\ \hline
    makebox & & \\ \hline
    getfield & & \\ \hline
    listhead & & \\ \hline
    listtail & & \\ \hline
    makeclosure & & \\ \hline
    makehlosure & & \\ \hline
    bin. operator bop & Performs bop & A := bop result, S.pop \\
\end{tabular}
\caption{SHADE instruction set}
\label{shade-bytecode}
\end{table}

\subsection{Compiling for the SHADE machine}

\begin{figure}
    $$ \interp{\irId\ x}_s^\gamma = \text{where\_is}\ (x, s, \gamma) $$
    $$ \interp{\irConst\ c} = \vmConst\ c $$
    $$ \interp{\irBox\ (\mathrm{tag}, v_1 \dots v_n)}_s ^ \gamma = \interp{v_n}_s^\gamma; \vmPush; \interp{v_{n-1}}_{s+1}^\gamma; \vmPush; \dots \interp{v_1}_{s+n-1}^\gamma; \vmMakeBox\ \mathrm{tag}, n $$
    $$ \underbrace{\interp{\irFun\ (x, body)}_s ^\gamma }_f = \text{fvs-to-stack} (s, \gamma, f); \vmMakeClosure (L, \mathrm{sz}) $$
    $$ \interp{\irHandler\ (v, es, f)} =  $$
    $$ \interp{\irVal\ v}_s ^ \gamma =  \interp{v}_s ^ \gamma $$
    $$ \interp{\irIf\ (v, e_1, e_2)}_s ^ \gamma = \vmBranchIfNot\ L_\text{false}; \interp{e_1}_s^\gamma; \vmJump\ L_\text{exit}; \vmLabel\ L_\text{false}; \interp{e_2}_s^\gamma; \vmLabel\ L_\text{exit} $$
    $$ \interp{\irLetIn\ (x, e_1, e_2)} =  $$
    $$ \interp{\irLetRecIn\ (f, x, body, e} =  $$
    $$ \interp{\irTopLet\ (x, e)} =  $$
    $$ \interp{\irTopLetRec\ (f, x, body)} =  $$
    $$ \interp{\irPerform\ (id, v)} = \interp{v}_s ^ \gamma; \vmThrow\ id $$
    $$ \interp{\irWithHandle\ (h, e)}_s ^ \gamma = \interp{h}_s ^ \gamma; \vmPush; \int$$
    $$ \interp{\irBinOp\ (\mathtt{op}, v_1, v_2} = \interp{v_1}_s ^ \gamma; \vmPush; \interp{v_2}_{s+1}; \text{instr-of}(\mathtt{op}) $$
    $$ \interp{\irFunApp\ (v_1, v_2)} = \interp{v_1}_s ^ \gamma; \vmPush; \interp{v_2}_{s+1} ^ \gamma $$
    $$ \interp{\irGetField\ (n, v)}_s ^ \gamma = \interp{v}_s ^ \gamma; \vmGetField\ n $$
    $$ \interp{\irListHead\ v}_s ^ \gamma = \interp{v}_s ^ \gamma; \vmListHead $$
    $$ \interp{\irListTail\ v}_s ^ \gamma = \interp{v}_s ^ \gamma ; \vmListTail $$

    \caption{Compilation from IR to SHADE bytecode}
    \label{compilation}
\end{figure}

\section{Software Engineering}

\subsection{Interfaces}
\subsection{Modules}
\begin{itemize}
    \item yoyo.ml
\end{itemize}

\subsection{Testing}
- regression testing
- tests

\subsection{Version control}
- via Git + Github
- Each new feature had its own branch + quick bugfix branch
- Refactor

\subsection{OPAM}
- The project is packaged as an OCaml package and is available under the name SHADE
- Can be installed with opam install shade

\subsection{Documentation}
- documentation is generated by odoc
- mli files and (** comments *)
- used this when I refactored code and to help structure my thoughts

\end{document}