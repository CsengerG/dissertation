\documentclass[class=article, crop=false]{standalone}
\usepackage[subpreambles=true]{standalone}
\usepackage{import}
\usepackage{ebproof}
\usepackage[utf8]{inputenc}
\usepackage{tikz}
\usepackage{hyperref}
\usepackage{amsmath}
\usepackage{amssymb}
\usepackage{listings}
\usepackage{verbatim}
\usepackage{a4wide}
\usepackage[super]{nth}
\usepackage{float}
\usepackage{subcaption}

\ifstandalone
\usepackage{color}
\usepackage{xcolor}
\usepackage{caption}
\usepackage{courier}

\lstdefinelanguage{efflang}
{
    % list of keywords
    morekeywords={
        let,
        perform,
        continue,
        val,
        effect,
        in,
        if, then, else,
        with, handle, handler,
        finally,
        match,
        exception,
        of
    },
    sensitive=false,
    morecomment=[s]{(*}{*)},
    morestring=[b]"
}

\lstdefinelanguage{scheme}
{
    % list of keywords
    morekeywords={
        define, call/cc, lambda
    },
    sensitive=false,
    morecomment=[s]{\#|}{|\#},
    morestring=[b]"
}

\lstset{
  basicstyle=\small\ttfamily, % Default font
  numberstyle=\small,          % Style of line numbers
  numbersep=5pt,              % Margin between line numbers and text
  tabsize=2,                  % Size of tabs
  extendedchars=true,
  breaklines=true,            % Lines will be wrapped
  keywordstyle=\color{red},
  frame=b,
  numbers=left,
  numberstyle=\footnotesize\color{gray},
  numbersep=10pt,
  captionpos=t,
  stringstyle=\color{purple!80!blue}\ttfamily, % Color of strings
  showspaces=false,
  showtabs=false,
  xleftmargin=17pt,
  framexleftmargin=17pt,
  framexbottommargin=4pt,
  showstringspaces=false
}

\DeclareCaptionFont{white}{\color{white}}
\DeclareCaptionFormat{listing}{\colorbox[cmyk]{0.43, 0.35, 0.35,0.01}{\parbox{\textwidth}{\hspace{15pt}#1#2#3}}}
\captionsetup[lstlisting]{format=listing,labelfont=white,textfont=white, singlelinecheck=false, margin=0pt, font={bf,footnotesize}}

\newcommand{\mylisting}[4]{%
\noindent
\begin{minipage}{\textwidth}
\lstinputlisting[
  language=#1,
  caption={#2},
  label=#3
  ]{#4}
\end{minipage}
  }

\lstset{language=efflang}
\fi

\ifstandalone
\usepackage{stmaryrd}
\usepackage{framed}

\renewcommand{\leadsto}{\rightsquigarrow}
\providecommand{\dmid}{\ \parallel \ }

\providecommand{\effFalse}{\mathbf{false}}
\providecommand{\effTrue}{\mathbf{true}}
\providecommand{\effLeft}{\mathbf{Left}\ }
\providecommand{\effRight}{\mathbf{Right\ }}
\providecommand{\effFun}{\mathbf{fun}\ }
\providecommand{\effRecFun}{\mathbf{recfun}\ }
\providecommand{\effHandler}{\mathbf{handler}\ }
\providecommand{\effVal}{\mathbf{val}\ }
\providecommand{\effWith}{\mathbf{with}\ }
\providecommand{\effHandle}{\ \mathbf{handle}\ }
\providecommand{\effIf}{\mathbf{if}\ }
\providecommand{\effThen}{\ \mathbf{then}\ }
\providecommand{\effElse}{\ \mathbf{else}\ }
\providecommand{\effAbsurd}{\mathbf{absurd}\ }
\providecommand{\effMatch}{\mathbf{match}\ }
\providecommand{\effLet}{\mathbf{let}\ }
\providecommand{\effIn}{\ \mathbf{in}\ }
\providecommand{\effRec}{\mathbf{rec}\ }
\providecommand{\effEffect}{\mathbf{effect}\ }
\providecommand{\effFinally}{\mathbf{finally}\ }
\providecommand{\effOp}{\mathtt{op}}
\providecommand{\effPerform}{\mathbf{perform}\ }
\providecommand{\tto}{\twoheadrightarrow}

\providecommand{\handlerType}{\Rightarrow}
\providecommand{\boolType}{\mathtt{bool}}
\providecommand{\unitType}{\mathtt{unit}}
\providecommand{\emptyType}{\mathtt{empty}}

\providecommand{\defEq}{\stackrel{\text{def}}{=}}

\providecommand{\cek}[1]{\langle #1 \rangle}
\providecommand{\secd}[1]{\langle #1 \rangle}
\providecommand{\shade}[1]{\langle #1 \rangle}

\providecommand{\irId}{\mathbf{Id}}
\providecommand{\irConst}{\mathbf{Const}}
\providecommand{\irBox}{\mathbf{Box}}
\providecommand{\irFun}{\mathbf{Fun}}
\providecommand{\irHandler}{\mathbf{Handler}}
\providecommand{\irVal}{\mathbf{Return}}
\providecommand{\irIf}{\mathbf{If}}
\providecommand{\irLetIn}{\mathbf{LetIn}}
\providecommand{\irLetRecIn}{\mathbf{LetRecIn}}
\providecommand{\irTopLet}{\mathbf{TopLet}}
\providecommand{\irTopLetRec}{\mathbf{TopLetRec}}
\providecommand{\irPerform}{\mathbf{Perform}}
\providecommand{\irWithHandle}{\mathbf{WithHandle}}
\providecommand{\irBinOp}{\mathbf{BinOp}}
\providecommand{\irFunApp}{\mathbf{FunApp}}
\providecommand{\irGetField}{\mathbf{GetField}}
\providecommand{\irListHead}{\mathbf{ListHead}}
\providecommand{\irListTail}{\mathbf{ListTail}}

\providecommand{\interp}[1]{\llbracket #1 \rrbracket}

\providecommand{\shUnit}{\mathbf{()}}
\providecommand{\shHalt}{\mathbf{halt}}
\providecommand{\shCast}{\mathbf{cast}}
\providecommand{\shRett}{\mathbf{ret2}}
\providecommand{\shApply}{\mathbf{apply}}
\providecommand{\shCastShadow}{\mathbf{castshadow}}
\providecommand{\shKillShadow}{\mathbf{killshadow}}
\providecommand{\shFin}{\mathbf{fin}}
\providecommand{\shThrow}{\mathbf{throw}}

\providecommand{\vmPush}{\textbf{push}}
\providecommand{\vmPop}{\textbf{pop}}
\providecommand{\vmAcc}[1]{\textbf{acc} #1}
\providecommand{\vmConst}[1]{\textbf{const} #1}
\providecommand{\vmHalt}{\textbf{halt}}
\providecommand{\vmJump}[1]{\textbf{jump }#1}
\providecommand{\vmLabel}{\textbf{label}}
\providecommand{\vmBranchIfNot}[1]{\textbf{branchifnot }#1}
\providecommand{\vmApply}{\textbf{apply}}
\providecommand{\vmRet}{\textbf{ret}}
\providecommand{\vmRett}{\textbf{ret2}}
\providecommand{\vmMakeBox}[2]{\textbf{makebox }#1, #2}
\providecommand{\vmGetField}[1]{\textbf{getfield }#1}
\providecommand{\vmListHead}{\textbf{listhead}}
\providecommand{\vmListTail}{\textbf{listtail}}
\providecommand{\vmMakeClosure}[2]{\textbf{makeclosure }#1, #2}
\providecommand{\vmMakeHlosure}[4]{\textbf{makehlosure }#1, #2, #3, #4}
\providecommand{\vmThrow}{\textbf{throw}}
\providecommand{\vmFin}{\textbf{fin}}
\providecommand{\vmCastShadow}{\textbf{castshadow}}
\providecommand{\vmKillShadow}{\textbf{killshadow}}

\providecommand{\hlosure}{\mathcal{H}}
\providecommand{\konts}{\mathcal{C}}

\newenvironment{myfigure}[4][0.75]{
    \def\mywidth{#1}
    \def\mycaption{#3}
    \def\mylabel{#4}
    \definecolor{shadecolor}{rgb}{0.95,0.95,0.95}

    \begin{figure}[#2]
    \centering
    \begin{minipage}{\mywidth\textwidth}
    \begin{shaded*}
}{
    \caption{\mycaption}
    \label{\mylabel}
    \end{shaded*}
    \end{minipage}
    \end{figure}
}
\fi

\begin{document}

Intro here
Eff vs CEK vs Shade VM vs Multicore Ocaml

\section{Exceptions}

\subsection{Results}

\section{State and I/O}

\subsection{Results}

\section{Non-determinism -- The N queens problem}

\subsection{Results}

\section{Concurrency -- How palindromic is a file?}

In the Introduction I talked about asynchronous APIs ...

Here I will show that there really is real world benefit from playing with continuations.
The following cooperative multithreading example schedules between many blocking tasks:

Imagine we want to decide....
We have a built-in call Compare i...

\subsection{Synchronous processing}

The naive approach would be something like this:

\begin{verbatim}
List.map (fun i -> perform (Compare i)) [1,2,3,...N]
\end{verbatim}

However, this will block sometimes as the bytes we need to compare are not immediately
available. Most of the time we will need to wait for the disk to seek if N is large enough.

\subsection{Asynchronous processing}

\begin{verbatim}
let create_blocking_process i =
    let rec try () =
        let is_same = perform (CompareAsync i) in
            if is_same = ~1 then
                perform (Yield ());
                try ()
            else
                is_same
    in try
;;
\end{verbatim}

The function \verb|create_blocking_process| creates a \verb|unit -> int| function which calls
our built-in Compare effect. This will look into a big file on the disk (I used a 1 GB file
here). The Compare effect is responsible for comparing the i-th and the (n-i)-th byte in a file.
If the two bytes are equal then the result of Compare is 1, else it is zero. If any of the bytes
is not yet available then Compare returns \verb|~1|, indicating that we need to retry later.

The resulting blocking process will hence return \verb|is_same| when the comparison could be done and it
will Yield otherwise, allowing an other process to make progress.

\subsubsection{Round-robin scheduler}

The handlers on listing X bring concurrency and state together.

\begin{verbatim}
type process = unit -> int;;

effect Spawn : process -->> int;;
effect Yield : unit -->> unit;;

effect QueueAdd : process -->> unit;;
effect QueueGet : unit -->> process;;

let spawn_process i = 
    perform (Spawn (create_blocking_process i))
;;

let fifo_queue = handler
| val x -> (fun _ -> x)
| effect (QueueAdd p) k -> fun () ->
| effect (QueueGet ()) k ->
| finally f -> f []
;;

let round_robin_scheduler =
    let rec round_robin () =
        handler
        | val x -> perform (
        | effect (Yield ()) k ->
            perform (QueueAdd k);
            perform (QueueGet ()
        | effect (Spawn p) k ->
            perform (QueueAdd k);
            with round_robin () handle
                p ()


let result =
    with fifo_queue handle
        with round_robin_scheduler handle
            map spawn_process [1,2,3...N]
;;

let palindromicity = fold add result;;
\end{verbatim}

\subsection{Results}

\end{document}