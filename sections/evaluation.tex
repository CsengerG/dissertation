\documentclass[class=article, crop=false]{standalone}
\usepackage[subpreambles=true]{standalone}
\usepackage{import}
\usepackage{ebproof}
\usepackage[utf8]{inputenc}
\usepackage{tikz}
\usepackage{hyperref}
\usepackage{amsmath}
\usepackage{amssymb}
\usepackage{listings}
\usepackage{verbatim}
\usepackage{a4wide}
\usepackage[super]{nth}
\usepackage{float}
\usepackage{subcaption}

\ifstandalone
\lstdefinelanguage{efflang}
{
    % list of keywords
    morekeywords={
        let,
        perform,
        continue,
        val,
        effect,
        in,
        if, then, else,
        with, handle, handler,
        finally,
        match
    },
    sensitive=false,
    morecomment=[s]{(*}{*)},
    morestring=[b]"
}

% Define Colors
\usepackage{color}
\definecolor{eclipseBlue}{RGB}{42,0.0,255}
\definecolor{eclipseGreen}{RGB}{63,127,95}
\definecolor{eclipsePurple}{RGB}{127,0,85}
\definecolor{bg}{RGB}{248,248,248}

\lstset{
    language={efflang},
    basicstyle=\small\ttfamily, % Global Code Style
    captionpos=b, % Position of the Caption (t for top, b for bottom)
    extendedchars=true, % Allows 256 instead of 128 ASCII characters
    tabsize=2, % number of spaces indented when discovering a tab 
    columns=fixed, % make all characters equal width
    keepspaces=true, % does not ignore spaces to fit width, convert tabs to spaces
    showstringspaces=false, % lets spaces in strings appear as real spaces
    breaklines=true, % wrap lines if they don't fit
    %frame=trbl, % draw a frame at the top, right, left and bottom of the listing
    %frameround=tttt, % make the frame round at all four corners
    framesep=4pt, % quarter circle size of the round corners
    numbers=left, % show line numbers at the left
    numberstyle=\tiny\ttfamily, % style of the line numbers
    commentstyle=\color{eclipseGreen}, % style of comments
    keywordstyle=\color{eclipsePurple}, % style of keywords
    stringstyle=\color{eclipseBlue}, % style of strings
    xleftmargin=.05\textwidth,
    framexleftmargin=17pt,
    backgroundcolor=\color{bg}
    %xrightmargin=.2\textwidth
}
\else
\lstdefinelanguage{efflang}
{
    % list of keywords
    morekeywords={
        let,
        perform,
        continue,
        val,
        effect,
        in,
        if, then, else,
        with, handle, handler,
        finally,
        match
    },
    sensitive=false,
    morecomment=[s]{(*}{*)},
    morestring=[b]"
}

% Define Colors
\usepackage{color}
\definecolor{eclipseBlue}{RGB}{42,0.0,255}
\definecolor{eclipseGreen}{RGB}{63,127,95}
\definecolor{eclipsePurple}{RGB}{127,0,85}
\definecolor{bg}{RGB}{248,248,248}

\lstset{
    language={efflang},
    basicstyle=\small\ttfamily, % Global Code Style
    captionpos=b, % Position of the Caption (t for top, b for bottom)
    extendedchars=true, % Allows 256 instead of 128 ASCII characters
    tabsize=2, % number of spaces indented when discovering a tab 
    columns=fixed, % make all characters equal width
    keepspaces=true, % does not ignore spaces to fit width, convert tabs to spaces
    showstringspaces=false, % lets spaces in strings appear as real spaces
    breaklines=true, % wrap lines if they don't fit
    %frame=trbl, % draw a frame at the top, right, left and bottom of the listing
    %frameround=tttt, % make the frame round at all four corners
    framesep=4pt, % quarter circle size of the round corners
    numbers=left, % show line numbers at the left
    numberstyle=\tiny\ttfamily, % style of the line numbers
    commentstyle=\color{eclipseGreen}, % style of comments
    keywordstyle=\color{eclipsePurple}, % style of keywords
    stringstyle=\color{eclipseBlue}, % style of strings
    xleftmargin=.05\textwidth,
    framexleftmargin=17pt,
    backgroundcolor=\color{bg}
    %xrightmargin=.2\textwidth
}
\fi

\ifstandalone
\renewcommand{\leadsto}{\rightsquigarrow}
\newcommand{\dmid}{\ \parallel \ }

\newcommand{\effFalse}{\mathbf{false}}
\newcommand{\effTrue}{\mathbf{true}}
\newcommand{\effLeft}{\mathbf{Left}\ }
\newcommand{\effRight}{\mathbf{Right\ }}
\newcommand{\effFun}{\mathbf{fun}\ }
\newcommand{\effHandler}{\mathbf{handler}\ }
\newcommand{\effVal}{\mathbf{val}\ }
\newcommand{\effWith}{\mathbf{with}\ }
\newcommand{\effHandle}{\ \mathbf{handle}\ }
\newcommand{\effIf}{\mathbf{if}\ }
\newcommand{\effThen}{\ \mathbf{then}\ }
\newcommand{\effElse}{\ \mathbf{else}\ }
\newcommand{\effAbsurd}{\mathbf{absurd}\ }
\newcommand{\effMatch}{\mathbf{match}\ }
\newcommand{\effLet}{\mathbf{let}\ }
\newcommand{\effIn}{\ \mathbf{in}\ }
\newcommand{\effRec}{\mathbf{rec}\ }
\newcommand{\effEffect}{\mathbf{effect}\ }
\newcommand{\effFinally}{\mathbf{finally}\ }
\newcommand{\effOp}{\mathtt{op}}
\newcommand{\effPerform}{\mathbf{perform}\ }
\newcommand{\tto}{\twoheadrightarrow}

\newcommand{\handlerType}{\Rightarrow}
\newcommand{\boolType}{\mathtt{bool}}
\newcommand{\unitType}{\mathtt{unit}}
\newcommand{\emptyType}{\mathtt{empty}}

\newcommand{\defEq}{\stackrel{\text{def}}{=}}

\newcommand{\cek}[1]{\langle #1 \rangle}
\newcommand{\secd}[1]{\langle #1 \rangle}
\fi

\begin{document}

Intro here
Eff vs CEK vs Shade VM vs Multicore Ocaml

\section{Exceptions}

\subsection{Results}

\section{State and I/O}

\subsection{Results}

\section{Non-determinism -- The N queens problem}

\subsection{Results}

\section{Concurrency -- How palindromic is a file?}

In the Introduction I talked about asynchronous APIs ...

Here I will show that there really is real world benefit from playing with continuations.
The following cooperative multithreading example schedules between many blocking tasks:

Imagine we want to decide....
We have a built-in call Compare i...

\subsection{Synchronous processing}

The naive approach would be something like this:

\begin{verbatim}
List.map (fun i -> perform (Compare i)) [1,2,3,...N]
\end{verbatim}

However, this will block sometimes as the bytes we need to compare are not immediately
available. Most of the time we will need to wait for the disk to seek if N is large enough.

\subsection{Asynchronous processing}

\begin{verbatim}
let create_blocking_process i =
    let rec try () =
        let is_same = perform (CompareAsync i) in
            if is_same = ~1 then
                perform (Yield ());
                try ()
            else
                is_same
    in try
;;
\end{verbatim}

The function \verb|create_blocking_process| creates a \verb|unit -> int| function which calls
our built-in Compare effect. This will look into a big file on the disk (I used a 1 GB file
here). The Compare effect is responsible for comparing the i-th and the (n-i)-th byte in a file.
If the two bytes are equal then the result of Compare is 1, else it is zero. If any of the bytes
is not yet available then Compare returns \verb|~1|, indicating that we need to retry later.

The resulting blocking process will hence return \verb|is_same| when the comparison could be done and it
will Yield otherwise, allowing an other process to make progress.

\subsubsection{Round-robin scheduler}

The handlers on listing X bring concurrency and state together.

\begin{verbatim}
type process = unit -> int;;

effect Spawn : process -->> int;;
effect Yield : unit -->> unit;;

effect QueueAdd : process -->> unit;;
effect QueueGet : unit -->> process;;

let spawn_process i = 
    perform (Spawn (create_blocking_process i))
;;

let fifo_queue = handler
| val x -> (fun _ -> x)
| effect (QueueAdd p) k -> fun () ->
| effect (QueueGet ()) k ->
| finally f -> f []
;;

let round_robin_scheduler =
    let rec round_robin () =
        handler
        | val x -> perform (
        | effect (Yield ()) k ->
            perform (QueueAdd k);
            perform (QueueGet ()
        | effect (Spawn p) k ->
            perform (QueueAdd k);
            with round_robin () handle
                p ()


let result =
    with fifo_queue handle
        with round_robin_scheduler handle
            map spawn_process [1,2,3...N]
;;

let palindromicity = fold add result;;
\end{verbatim}

\subsection{Results}

\end{document}